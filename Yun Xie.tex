%!TEX encoding = UTF-8 Unicode
%%%%%%%%%%%%%%%%%%%%%%%%%%%%%%%%%%%%%%%%%%%%%%%%%%%%%%%%%%%%%%%%%%%%%%%%%%%
%%%%%%%%%%%%%%%%%%%%%%%%%%%%%%%%%%%%%%%%%%%%%%%%%%%%%%%%%%%%%%%%%%%%%%%%%%%
%%                                                                       %%
%%                    《逻辑学研究》中文论文模板                         %%
%%                                                                       %%
%%              中山大学逻辑与认知研究所逻辑学研究编辑部                 %%
%%                                                                       %%
%%                             Ver 1.3                                   %%
%%                                                                       %%
%%        You can modify it and distribute it freely    2013.03.21       %%
%%                                                                       %%
%%%%%%%%%%%%%%%%%%%%%%%%%%%%%%%%%%%%%%%%%%%%%%%%%%%%%%%%%%%%%%%%%%%%%%%%%%%

%-------------------------------------------------------------------------%
%
%  请以第一作者全拼为文件名另存此文档(后缀名仍为.tex),与 SLCN.sty 保存
%
%  在同一个文件夹中。你可能需要取消某些行首的注释符以添加需要的内容。
%  使用XeLaTex编译
%-------------------------------------------------------------------------%

%=========================================================================%
%                        一、编辑处理部分
%
%                  *** 作者请直接跳至第二部分 ***
%=========================================================================%

%-------------------------------------------------------------------------%
%    1.1 设定纸张大小、正文字体大小
%-------------------------------------------------------------------------%
\documentclass[b5paper,10.5pt,onecolumn,twoside,UTF8]{article}
\usepackage{SLCN} 
\usepackage[all]{xy}                                                        % 加载版面格式
\setmainfont{Times New Roman}
\setCJKsansfont[FakeBold=2]{KaiTi}

%-------------------------------------------------------------------------%
%    1.2 填入卷期号、出版年月
%-------------------------------------------------------------------------%
\newcommand{\myvolnumber}{12}                                              % 输入卷号
\newcommand{\myissnumber}{5}                                              % 输入当年期号
\newcommand{\mypubyear}{2019}                                             % 输入出版年份

%-------------------------------------------------------------------------%
%    1.3 填入起止页码、页数
%-------------------------------------------------------------------------%
\newcommand{\myfirstpage}{98}                                              % 输入起始页码
\newcommand{\mylastpage}{108}                                              % 输入终止页码
\newcommand{\mypages}{11}                                                 % 输入页数

%-------------------------------------------------------------------------%
%    1.4 填入收稿日期、修改稿日期
%-------------------------------------------------------------------------%
\newcommand{\receiveddate}{2019-05-03} %2018-12-04                                   % 输入本文收稿日期
\newcommand{\revisiondate}{null}                                          % 预置修订日期为空,勿改此行
%\renewcommand{\revisiondate}{xxxx-xx-xx}                                 % 输入修订日期(若有)并取消该行注释

%-------------------------------------------------------------------------%
%    1.5 填入作者、单位英译名
%-------------------------------------------------------------------------%
\newcommand{\mysecondauthorEN}{null}                                      % 预置第二作者为空,请不要修改此行
\newcommand{\mythirdauthorEN}{null}                                       % 预置第三作者为空,请不要修改此行
\newcommand{\myfourthauthorEN}{null}
\newcommand{\myfifthauthorEN}{null}

\newcommand{\myfirstauthorEN}                                             % 请输入第一作者姓名
{Yun Xie}

\newcommand{\myfirstaffiliationEN}{%                                       % 请输入第一作者单位
{Institute of Logic and Cognition, Sun Yat-sen University
\\ \hspace{1em}\phantom{\myfirstauthorEN}\hspace{1em} Department of Philosophy, Sun Yat-sen University}
}

%\renewcommand{\mysecondauthorEN}{Second Author}                               % 若需要,请输入第二作者姓名,并取消该行注释
\newcommand{\mysecondaffiliationEN}{                                      % 若需要,请输入第二作者单位
\small{\it affiliation 2} \\
\small{\it another affiliation}
}

%\renewcommand{\mythirdauthorEN}{Third Author}                                % 若需要,请输入第三作者姓名,并取消该行注释
\newcommand{\mythirdaffiliationEN}{                                       % 若需要,请输入第三作者单位
\small{\it affiliation 3} \\
\small{\it another affiliation}
}

%\renewcommand{\myfourthauthorEN}{Fourth Author}
\newcommand{\myfourthaffiliationEN}{
\small{\it affiliation 4}
}


%\renewcommand{\myfifthauthorEN}{Fifth Author}
\newcommand{\myfifthaffiliationEN}{
\small{\it Affiliation 5}
}

%-------------------------------------------------------------------------%
%    1.6 填入文章类型
%  (original, bookreview, conferencereport)默认为original
%-------------------------------------------------------------------------%
\newcommand{\myarticletype}
{original}

\newcommand{\reviewbooktitle}                                             % 若文章为书评,请输入所评书的出版信息
{The information of the book reviewed by the author}

\newcommand{\reviewbooktitleEN}{null}                                     % 预置所评书的中译版为空,请勿修改此F行
%\renewcommand{\reviewbooktitleEN}{中译版信息}                            % 若书有中译版,请输入中译版信息并取消该行注释

%-------------------------------------------------------------------------%
%    1.7 填入责任编辑
%-------------------------------------------------------------------------%
\newcommand{\myeditor}{
{\bf (责任编辑:曾欢)}
}

%-------------------------------------------------------------------------%
%    1.8 缺省设置
%-------------------------------------------------------------------------%
\newcommand{\mysecondauthor}{null}                                        % 预置第二作者为空,请不要修改此行
\newcommand{\mythirdauthor}{null}                                         % 预置第三作者为空,请不要修改此行
\newcommand{\mygrants}{null}                                              % 预置“项目资助”为空,请不要修改此行
\newcommand{\mythanks}{null}                                              % 预置“致谢”为空,请不要修改此行

\usepackage{graphicx}
%=========================================================================%
%
%                        二、作者写作部分
%
%=========================================================================%

%-------------------------------------------------------------------------%
%    2.1 请填入论文题目、作者姓名、单位、电子邮箱
%-------------------------------------------------------------------------%
\newcommand{\mytitle}                                                     % 请输入论文题目
{权衡论证:一种语用论辩学的分析}

\newcommand{\myrunningtitle}                                              % 请输入用于页眉的标题(可能需要缩短原来的标题)
{权衡论证:一种语用论辩学的分析}

\newcommand{\myfirstauthor}                                               % 请输入第一作者姓名
{\kaishu  谢耘}

\newcommand{\myfirstaffiliation}                                          % 请输入第一作者单位,多个单位用 \\\small 分隔
{中山大学逻辑与认知研究所、中山大学哲学系}

\newcommand{\myfirstemail}                                                % 请输入第一作者电子邮箱
{xieyun6@mail.sysu.edu.cn}

%\renewcommand{\mysecondauthor}{\kaishu 第二作者}                                 % 若需要,请输入第二作者姓名,并取消该行注释

\newcommand{\mysecondaffiliation}                                         % 若需要,请输入第二作者单位,多个单位用\\\small分隔
{\small 单位2(到系所)}

\newcommand{\mysecondemail}                                               % 若需要,请输入第二作者邮箱
{\small xxxx@xxxx.xxx}

%\renewcommand{\mythirdauthor}{\kaishu 第三作者}                                  % 若需要,请输入第三作者姓名,并取消该行注释

\newcommand{\mythirdaffiliation}                                          % 若需要,请输入第三作者单位,多个单位用\\small分隔
{\small 单位3(到系所)}

\newcommand{\mythirdemail}                                                % 若需要,请输入第三作者邮箱
{\small xxxx@xxxx.xxx}

\newcommand{\myfourthauthor}{null}

%\renewcommand{\myfourthauthor}{\kaishu 第四作者}
\newcommand{\myfourthaffiliation}{\small 单位四}

\newcommand{\myfourthemail}
{\small xxxx@xxxx.xxx}

\newcommand{\myfifthauthor}{null}

%\renewcommand{\myfifthauthor}{\kaishu 第五作者}
\newcommand{\myfifthaffiliation}{\small 单位五}

\newcommand{\myfifthemail}
{\small xxxx@xxxx.xxx}
%-------------------------------------------------------------------------%
%    2.2 请填入项目资助、致谢(选填)
%------------------------------------------------------------------------%
\renewcommand{\mygrants}{2018年度国家社科基金重大项目“广义论证的理论与实验”(18ZDA033)。}                  % 若需要,请输入项目名称及批号,并取消该行注释,多个项目以中文逗号分隔
%\renewcommand{\mythanks}{~}                          % 若需要,请输入致谢内容,并取消该行注释

%-------------------------------------------------------------------------%
%    2.3 请填入中文摘要、关键词
%-------------------------------------------------------------------------%
\newcommand{\myabstract}                                                  % 请在下面输入中文摘要
{由于权衡论证中同时引用了正反两方面的理由来证明一个观点,它被视为一种特殊的论证类型,并得到学界的持续关注和探讨。非形式逻辑学者认为,权衡论证对应着一个“将正反两方面理由加以权衡后得出结论”的特殊机制,因而需要通过增加“平衡考虑前提”来对其进行逻辑重构。这一做法强调反面理由的逻辑功能,但却带来一系列的理论问题。借助语用论辩学的理论工具,可以将权衡论证解析为一种策略操控的特定模式,这一分析不仅能很好地揭示权衡论证的特点和机制,而且还展现出理论上的简洁性和优越性。权衡论证中引述反面理由的目的,并不是为结论的证成提供逻辑依据,而是为增加说服效果而采取的修辞策略。进而,要刻画权衡论证的基本特性,其关键在于阐明其中所运用的特殊语言表达技巧,以及它如何增进了论证对于听众的说服效果。}

\newcommand{\mykeywords}                                                  % 请在下面输入中文关键词,以中文分号分隔
{权衡论证;非形式逻辑;语用论辩学;逻辑重构;修辞学}

%-------------------------------------------------------------------------%
%    2.4 请填入英文标题、摘要(默认为null)
%-------------------------------------------------------------------------%
\newcommand{\mytitleEN}                                                   % 请输入英文标题
{Balance-of-Considerations Argument:\\ A Pragma-Dialectical Analysis}

\newcommand{\myabstractEN}                                                % 请输入英文摘要
{In balance-of-considerations arguments both the positive reasons and some counter-considerations are provided, implying that the conclusion is reached by the way of weighing and balancing. Informal logicians propose to reconstruct balance-of-considerations arguments by supplementing for them an on-balance premise. This approach emphasizes the logical function of counter-considerations, but interprets the arguer in an uncharitable way. Taking a Pragma-Dialectical perspective, balance-of-considerations arguments can be analyzed as a mode of strategic maneuvering. The presence of counter-considerations is understood as arguer's rhetorical strategy to increase the persuasive effect, rather than to enhance its justificatory power. A Pragma-dialectical analysis of balance-of-considerations arguments can well explain its special structure and mechanism, and also manifest some theoretical advantages over the logical perspective. }

%-------------------------------------------------------------------------%
%    2.5 预置宏包和自定义命令(你可以补充需要的宏包和自定义命令)
%-------------------------------------------------------------------------%
\usepackage{pifont}
\usepackage[pdfborder=0,CJKbookmarks=true]{hyperref}  % 使用内部超链接,其中第二个选项用于支持中文书签

\addbibresource{Yun Xie.bib}				%填入参考文献库文件 .bib

\usepackage{indentfirst}

\makeatletter
\renewenvironment{quotation}
               {\renewcommand{\baselinestretch}{1.2} \it
               \list{}{\listparindent 2em%
               \rightmargin=0em \leftmargin=2em
                        \itemindent    \listparindent
                        \rightmargin   \leftmargin
                        \parsep        \z@ \@plus\p@}%
                \item\relax}
               {\endlist}
               
\renewenvironment{quote}
               {\renewcommand{\baselinestretch}{1.2} \it
               \list{}{\listparindent 0em%
               \rightmargin=0em \leftmargin=2em
                        \itemindent    \listparindent
                        \rightmargin   \leftmargin
                        \parsep        \z@ \@plus\p@}%
                \item\relax}
               {\endlist}
               
\makeatother 
  
  
\newcommand{\xxskip}{\hspace{0.2em}}                                       
\newcommand{\xskip}{\hspace{0.2em}}                                       
%-------------------------------------------------------------------------%
%    2.6 打印标题页信息(作者请忽略此部分)
%-------------------------------------------------------------------------%
\begin{document}
\begingroup                                                               % 以下使得收稿日期以不加标记的脚注出现
\makeatletter
\let\@makefnmark\relax\footnotetext{%
  \ifthenelse{\equal{\revisiondate}{null}}{{\bf 收稿日期:}\receiveddate}{
    {\bf 收稿日期:}\receiveddate;{\bf 修订日期:}\revisiondate
  }
}
\makeatother
\endgroup
\printtitlepage                                                           % 打印标题、作者、摘要等信息
%-------------------------------------------------------------------------%
%    2.7 正文内容从这里开始
%-------------------------------------------------------------------------%
%\vspace{-1em}

\section{“权衡论证”的概念}

\vspace{-0.5em}

在现实生活的论证实践中,为了表明自己所持的观点正确,或者说服他人对之加以接受,我们会明确引述自己所认可的一些理由,来证明所持观点的合理性。通常情况下,我们是在论证中给出一些能够支持观点成立的“正面理由”(positive reasons),来直接奠定所持观点的合理性。但有些时候,我们也会在论证中引述一些可能导致观点不成立的“反面理由”(counter-considerations),来表明该观点的成立可以经受住它们的考验,并且仍然能够得到辩护。同时引用正反两方面的理由来论证一个观点,这表明了该观点的合理性是经过对正反两方面理由加以比较和权衡之后才得到的。在很大程度上,该观点之所以成立,可以被看作是因为正面理由强过了反面理由。在现实生活中,这样一种论证也通常被称为“权衡论证”(pro and con argument/balance-of-considerations argument),相比于仅提供正面理由的论证方式,它在很多情况下都能达到更好的说服效果(参见\parencite{a9})。

藉由道德哲学家威尔曼(C. Wellman)的《挑战与回应:伦理学中的证成》(1971)一书,“权衡论证”开始正式进入当代论证研究的理论领域,并随之得到了持续的关注与探讨。在该书中,威尔曼提出了“联导论证”(conductive argument)这一新论证类型,并且把“权衡论证”明确作为了该类论证的第三种模式(pattern):

\begin{quotation}
联导论证的第三种模式是这样一种论证方式:其结论是同时从正面理由和反面理由中得出的。在此种论证模式当中,既给出了与结论相关的正面理由,也包含了与之相关的反面理由。……\textup{[}例如\textup{]}尽管你家草坪需要修剪了,但你还是应该带你儿子去看电影,因为那部电影非常适合小孩,而且明天就要下线了。\rm(\parencite{a13},第57页)
\end{quotation}

从上述例子来看,该论证既给出了正面理由(“那部电影非常适合小孩”和“那部电影明天就要下线了”),也给出了反面理由(“你家草坪需要修剪”),而其结论(“你应该带你儿子去看电影”)则是在同时考虑了这两方面理由之后得出的。在威尔曼看来,该结论之所以在明确存在一个反面理由时仍然能够得以成立,显然是因为正面理由对结论的支持力强于反面理由对结论的破坏力。因而,他认为联导论证对应了一种特殊的证成方式,是一种既非演绎、又非归纳的全新论证类型。随后,经由戈维尔(T. Govier)的引介(\parencite{a5}),“联导论证”、尤其是“权衡论证”逐渐成为了学界探讨“论证类型学”(topology of argument)的一个核心议题。进而,非形式逻辑学者针对权衡论证展开了大量探讨,他们也力图将权衡论证阐释为“一类被忽略的可废止论证”,并强调它是“第三类论证”的一个典型样本(参见\parencite{a3,a6})。

近年来,权衡论证也成为当代论证研究领域的一个重要主题。随着更多学者的关注和探讨的不断深入,与之相关的理论争议也开始出现,并进而形成了一场关于权衡论证独特性和重要性的激烈争论。多数非形式逻辑学者坚持认为,在权衡论证中体现着独特的证成机制和复杂的论证结构,因而其分析与评估都需要发展全新的方法和理论。然而,另一些论证学者却对之表示强烈质疑和反对,他们深信这一理论企图是误入歧途(参见\parencite{a10,a14}),甚至认为“权衡论证”这一概念本身就是成问题的,它展现着对论证实践的错误认知(\parencite{a1})。时至今日,双方的争论仍在持续,相互之间的批评与回应还在不断涌现(\parencite{a2,a7,a11})。本文将从“语用论辩学”(The Pragma-Dialectics)的理论视角来对权衡论证加以分析和探讨,以期为当前学界的相关讨论贡献一种新的观点,同时,也为其理论争议的解决提供一种可能的方式。后文将首先对非形式逻辑学者关于权衡论证的结构分析加以探讨,并揭示其理论问题。随后,运用语用论辩学的“策略操控”理论,来对权衡论证的特点和机制加以阐明。以之为基础,本文进一步对当代论证研究的逻辑学进路作出理论反思。

\vspace{-0.5em}

\section{权衡论证的逻辑重构}

\vspace{-0.5em}

许多非形式逻辑学者认为,在权衡论证中论证者通过明确引述正反两方面的理由,来试图向听众表明结论是经由权衡所得,因为正面理由能够胜过反面理由,保证了结论仍然成立。与此相应,这一内在的“权衡机制”(weighing and balancing)也即是刻画权衡论证的关键,因而,要恰当解析权衡论证的逻辑结构,就需要合理地重构其中的这一权衡机制。依他们之见,虽然权衡论证在表达形态上只包括正面理由、反面理由和结论,但是,论证者以“尽管”“纵使”这类转折性关联词来引导反面理由,这一做法明确降低了反面理由的重要性,从而也暗示出正面理由对结论的支持力强于反面理由所具有的破坏力。因而,这一特定论证建构方式也即意味着,论证者在权衡论证中实际上还使用了一个代表这一强弱对比关系的特殊前提——“平衡考虑前提”(on-balance premise)。因此,他们试图通过补充平衡考虑前提的方式来重构权衡论证的逻辑结构(\parencite{a2},第124页;\parencite{a7},第274页):

\begin{itemize}
\item[]前提1:理由$a$,$b$,$c$,……支持结论$p$,

\item[]前提2:理由$w$,$x$,$y$,……支持结论 并非$p$,

\item[]前提3:理由$a$,$b$,$c$,……强于理由$w$,$x$,$y$,……(或者,情况相反)

\item[]所以,

\item[]结  论:$p$(或者并非$p$) 
\end{itemize}

与此类似,汉森(H. Hansen)还提出了一个更为复杂的重构方法,其中将权衡论证解析为包含着两个子论证的复合论证(\parencite{a8},第39页):

\begin{itemize}

\item[]子论证1:

\item[]前提1:理由1(支持结论$K$)。

\item[]…… 

\item[]前提$n$:理由$n$(支持结论$K$)。

\item[]前提$n+1$:理由$1$到$n$的组合整体上强于反面理由$CC_1$, $CC_2$, …$CC_n$组合。

\item[]所以,

\item[]结 论:尽管有$CC_1$, $CC_2$, …$CC_n$,$p$仍然成立。

\end{itemize}

\begin{itemize}

\item[]子论证2:

\item[]前 提:尽管有$CC_1$, $CC_2$, …$CC_n$,$p$仍然成立。

\item[]所以,

\item[]结 论:$p$成立。

\end{itemize}

然而,对于权衡论证作此种方式的逻辑重构,不仅使得其结构复杂化,而且还随之带来一系列理论问题。首先,“反面理由”并不支持结论成立,而是让人质疑结论的合理性,由此而言,将它们作为论证的“前提”,这显然是反直觉的,因为一直以来“前提”都是被理解和界定为那些能够支持结论成立的理由。因此,这一重构方式不仅要求着彻底变革传统的“前提”定义,甚至还可能导致修改既定的“论证”概念。而这两个概念都是论证研究中最为基本的核心概念,所以,这一重构方式是否得当、其理论后果能否接受,就成为了分歧严重的理论议题。其次,从权衡论证的表述来看,正面理由与反面理由都是独立地与结论相关,但在逻辑重构之后它们却变成了经由平衡考虑而共同得出结论。换言之,从权衡论证的实际形态来看,它呈现出“收敛结构”(convergent structure),而在重构之后却变成为“组合结构”(linked structure)。由此,权衡论证的结构到底是何种类型,这也成为一个争议问题,至今仍然难有定论(参见\parencite{a16})。

实际上,上述逻辑重构方式的背后还隐藏着一个基本的理论假定,那就是反面理由在权衡论证中必然起着前提的作用。也就是说,论证者之所以在权衡论证中引述反面理由,是认为它们与结论的证成具有逻辑关联,进而将它们用作前提来表明结论的合理性。显然,在建构权衡论证时,论证者对于反面理由的明确提及,这必定是有意为之。然而,要揭示这一做法的真实交际意图,则需要借助相关的语用分析,而不是简单的理论预设。在权衡论证中,反面理由是由“尽管”、“纵使”这类转折性关联词来引导,这一特定的表达方式既承认了反面理由的合法性,同时又强化了正反面理由之间的不均衡对比关系(即正面理由强于反面理由)。从交际的角度来看,论证者可以由此向听众表明,他完全知晓反面理由的存在并对它们加以过考虑,同时,他也可能随之暗示出“它们已经被正面理由推翻(outweighed)”(\parencite{a1},第247页)。进而,在交际效果上,论证者可以使自己显得更为客观和公允,也使其论证显得更为全面和可靠,从而更易于说服其听众。但是,严格来讲,从论证者关于反面理由的这一特定表达方式中,我们并不能明确判定他也将反面理由当作了“前提”,并用来和正面理由一起证明其结论的合理性。

\indent 更明确而言,在建构权衡论证时,论证者以一种特定的方式来提及反面理由,这虽然传递着明确的“会话含义”(conversational implicature),具有相应的交际意图,但却并不必然意味着他同时认为反面理由在该论证中与其结论的证立具有逻辑关联。与此相应,权衡论证中的反面理由并不必然要求一种以“结论证成”为核心的逻辑重构,它也可以对应于某种以“交际效果”为导向的修辞分析。简言之,论证者在权衡论证中提及反面理由,这完全可以被理解成不是在为证成结论提供逻辑依据,而是为了增加说服效果而采取的修辞策略。而这一解读方式既可以为权衡论证的解析提供一种新的途径,同时又能有效避免上述逻辑重构方式所带来的理论困境。

\vspace{-0.5em}

\section{“策略操控”的概念及其理论工具}

\vspace{-0.5em}

把对反面理由的提及视为增加说服效果的修辞策略,这实际上也就将权衡论证解读成了一种兼顾“证成结论”和“达成说服”的双重努力:论证者引述所认可的正面理由,来表明结论得以成立,同时,以特定方式提及部分反面理由,来获取最佳的说服效果。此种结合多重维度来解析论证行为的方式,正好与“语用论辩学”理论的最新发展相契合。因而,我们可以尝试运用该理论中的方法和工具,来对权衡论证做一个新的分析和探讨。

语用论辩学理论由范爱默伦(F.~van Eemeren)和荷罗顿道斯特(R.~Grootendorst)共同提出,并在上世纪末形成了其理论体系的标准形态。“语用论辩学标准理论”(the pragma-dialectical standard theory)主要采取了“论辩术”的理论进路,它将论证视作为“旨在基于不同意见的价值来解决意见分歧”,并进而提出了一个四阶段的“批判性讨论”基本模型,来对论证实践的合理性加以分析和评估。在过去十几年中,随着该理论不断在实践领域中得到应用和拓展,它也逐渐发展为更全面的“语用论辩学扩展理论”(the extended pragma-dialectical theory)。扩展理论增加了“修辞学”的研究维度,它强调真实论证者同时具有“保证论证合理”和“取得说服效果”的双重目标,并力图通过“策略操控”(strategic maneuvering)的手段来协调这两者的同时实现(参见\parencite{a15})。“策略操控”是语用论辩学扩展理论中最核心的一个概念,它指的是“在论证性会话的各个步骤中,论证者为了在合理性(reasonableness)与取效性(effectiveness)之间达到某种均衡而做出的那些努力”。(\parencite{a4},第40页)。通过发展与之相关的分析视角和理论工具,语用论辩学能够很好地解释真实论证建构过程中的“策略筹划”(strategic design)现象,从而更清楚地说明论证者在实施其论证行为时是如何在“保持合理”的同时,还力图最大程度地实现其“想要取效”的目标。

依照语用论辩学理论,策略操控会发生在批判性讨论的每一个阶段,甚至可能体现于每一个论证行为当中,对之加以分析则需要从三个方面入手。首先,每一个策略操控的模式,都涉及到论证者对于“潜在理由”(topical potential)的一次选择,也即是说,论证者对于自己在当时论证情境中能够加以应用的论证素材和手段,进行了一次适当的取舍。阐明这一选择和取舍的方式,可以揭示论证者的论证建构策略和论证行为动机。其次,每一个策略操控的模式,都涉及到论证者对于“听众需求”(audience demand)的一次主动适应,也即是说,论证者对于自己所面对的那个听众对象,以及他们在论证活动中所具有的基本诉求
%,进行依照语用论辩学理论,策略操控会发生在批判性讨论的每一个阶段,甚至可能体现于每一个论证行为当中,对之加以分析则需要从三个方面入手。首先,每一个策略操控的模式,都涉及到论证者对于“潜在理由”(topical potential)的一次选择,也即是说,论证者对于自己在当时论证情境中能够加以应用的论证素材和手段,进行了一次适当的取舍。阐明这一选择和取舍的方式,可以揭示论证者的论证建构策略和论证行为动机。其次,每一个策略操控的模式,都涉及到论证者对于“听众需求”(audience demand)的一次主动适应,也即是说,论证者对于自己所面对的那个听众对象,以及他们在论证活动中所具有的基本诉求
,进行过积极了解,并且尽量通过改变自己来对之加以适应。阐明这一适应的方式,可以揭示论证者迎合听众、谋求取效的特定方法。第三,每一个策略操控的模式,都涉及到论证者对于某种(或某些)特定“表达方式”(presentational device)的专门应用,也即是说,论证者对于如何呈现自己所选定的论证素材,以及如何表达自己所建构的论证,进行了精心的设计,采用了特定的手段。阐明这一特定的论证表达方式,可以揭示论证者实施其论证行为的基本技巧,并解释其可能实现的最佳说服效果(\parencite{a4},第93--94页)。

\vspace{-0.5em}

\section{权衡论证作为策略操控的一种特定模式}

\vspace{-0.5em}

作为语用论辩学分析论证行为的基本理论工具,“策略操控”也为解析权衡论证提供了一个全新途径。在批判性讨论的不同阶段,策略操控会有不尽相同的模式,而权衡论证显然是批判性讨论中“论辩阶段”策略操控的一种特定模式。更明确而言,权衡论证是在论辩阶段中可以被允许的一种行为步骤,它以提供正面理由来保证其论证具有合理性,并通过提及反面理由来促进其修辞目标的有效实现。具体来看,在权衡论证中论证者明确提及了反面理由,但与此同时,他又仍然确定论证的结论是成立的,而且也并未对之做出任何形式的限定。由此而言,论证者所采用的特定策略操控手段即表现为:在其论证中刻意地以一种特定的方式来提及少量影响结论成立的反面理由,而该提及方式又正好使得这些反面理由显得无足轻重或没有意义。进而,要对权衡论证的特性做出恰当分析,也就意味着要对这一特殊论证建构方式加以清晰解释,并阐明它如何与权衡论证的说服效果相关联。

首先,每一个权衡论证都涉及到论证者对于“潜在理由”的一次审慎选择。显然,在建构权衡论证时,论证者选择了他认为最有力的一个或多个支持性理由来证明其结论成立。但更重要地是,他同时还专门挑选了少量的反面理由来加以提及。并且,论证者对于反面理由的选择无疑是经过精心思考的。一方面,被提及的反面理由在数量和内容上都经过了审慎考虑。与一个论证相关联的反面理由通常可以有很多,有一些反面理由是针对前提的,它们表明了论证的前提可能是成问题的;还有一些反面理由是针对结论的,它们表明了论证的结论可能无法成立。但是,在权衡论证中论证者只会提及非常少量的一个或几个反面理由,而且它们通常只是针对结论、而不是针对前提的反面理由。另一方面,被提及的反面理由也需要具有易于被识别和认可的理论价值。事实上,我们可以发现,在建构权衡论证时,论证者选择提及的都是那些明显的、被认为值得考虑的反面理由,而且,它们通常还会是听众已经知晓或认可的反面理由。与此相应,当这些经过审慎选择的反面理由出现在论证中时,听众就会从中得到一个基本的印象:论证者似乎已经对论证的主题做出过全面和深入的思考,因此,他才能在论述其结论时从正反两个方面来加以考察,并进而得出了一个恰当的结论。与此同时,论证者个人也会由此而显得更像是一个品性诚实、态度客观和思想开放的人。而这一效果无疑也会增加听众对于论证者的信赖程度,进而使他们更容易被其所说服。

其次,每一个权衡论证都涉及到论证者对于“听众需求”的一次主动适应。一般而言,当论证者需要通过做出论证来向其听众表明某个结论成立时,也即意味着该结论在当时还没有获得听众的认可,他们对结论的成立正持有存疑的态度。而且,在很多情况下,听众之所以存疑也就是因为他们自己持有或已经知晓了某些反面理由。然而,如前所析,论证者在权衡论证中选择提及的那些反面理由,其存在通常为听众所知晓、其价值也需要为听众所认可。通过明确提及这些听众所熟悉的反面理由,论证者也就适当地表现出了他对于听众存疑态度的了解和尊重,并且,也显示出他对于这些反面理由之价值的承认与重视。这也即展现了论证者对于听众所持怀疑立场的一种让步,从而也由此在他们之间建立起某种“共通”(communion)之处。同时,提及听众所熟悉的反面理由也使得论证本身在表述上变得更为中立,在语气上更为缓和,进而也能减弱听众本来因其怀疑态度而可能具有的对抗倾向,这同样使得权衡论证更容易得到听众认可和接受,达到其理想的说服效果。

第三,每一个权衡论证都涉及到论证者对于一种特定“表达方式”的巧妙应用。实际上,在建构论证时,我们可以有很多种方式来引入反面理由。在很多情况下,我们在论述观点时不仅提及反面理由,而且还会对其合理性和相关性加以讨论。从认识论的角度而言,在证成一个观点时去考虑那些与之相关的反面理由,更多是为了要对它们加以检视,甚至是为了反驳它们。然而,论证者在权衡论证中却使用了一种非常特殊的方式来提及反面理由。一方面,反面理由仅仅是被简单地提及,而完全没有对之作任何进一步分析和反驳;另一方面,它们都是通过“尽管”“纵使”一类的关联词来加以引述,从而被直接与正面理由并列,以形成一个明显的转折关系。这不仅构造出正反面理由之间的一种鲜明对比,而且还使得双方在重要性上形成巨大落差。正面理由的价值得到了特别强调,而反面理由的意义则变得微不足道。藉由这一对比,论证者无疑向听众传递了一个显而易见的预设,即“正面理由强于反面理由”。由此,结论在反面理由面前仍然可以得到辩护,它们并不能对之产生影响。相应地,听众明显也会被引导而认为这些反面理由实际上弱于正面理由,或者,它们已经(基于某些其它原因而)失去了效力。换言之,虽然确实存在被提及的那些反面理由,但是我们却有(相应的或好的)理由和信心,去坚持结论依然成立。由此可见,权衡论证中反面理由的提及方式特殊而精妙,它使得论证者能够在未曾明言、甚至也未提供任何理据支持的情况下,就使得听众自己直接认识到、甚至于主动接受了“正面理由强于反面理由”这一看法。也正是经由这一过程,论证者向听众暗示了某种权衡机制,从而也使得其结论变成了一个权衡所得的公允结果。这一做法无疑能够很好地增强论证本身的说服效力,并且,尤其当听众对于反面理由只是知悉却没有深入把握时,其效果会特别明显。

\vspace{-0.5em}

\section{从逻辑重构到修辞分析}

\vspace{-0.5em}

可见,借助语用论辩学中策略操控的理论概念和分析方法,我们可以对权衡论证的机制和效果做出一个适当的说明。而且,这一解读方式既没有对权衡论证加以某种复杂重构,也没有带来触及论证理论核心概念的争议问题。换言之,将权衡论证作为一种策略操控模式,这一分析方法不仅具有相当程度的解释力,而且还展现出理论的简洁性。由此而言,权衡论证的解析也我们提供了一个理论契机,来对当代论证研究中的不同进路做一些比较和反思。 

在权衡论证中,论证者通过提及反面理由来使结论展现为权衡的结果,从而增加了其论证的说服效果。而这一切又只是通过一个非常简单的语言手段(以“尽管”“纵使”这类转折性关联词来引导反面理由)来加以实现的。与此相应,可以说权衡论证中实际上凸显着一个亟待阐明的吊诡之处:论证本身取得了因“权衡”而带来的积极效果,但却又并没有对其所依赖的那个权衡过程给出任何直接说明;论证的结论变成了一个经由权衡得出的公允结果,但论证中却没有对何以得出该结果(即正面理由为何胜过反面理由)给出任何的明确依据。换言之,对于其自身至为关键、最不可或缺的要素成分,在权衡论证中竟然都被加以了省略。然而,这种在表达形态上的(极端)“不完全性”,如何能够与其所具有的(甚至是相对而言更好的)“说服力”相协调?

非形式逻辑学者无疑是从逻辑学的理论进路来回答这一问题。从逻辑的观点来看,论证的好坏取决于前提是否能充分地证明结论的合理性,相应地,论证的说服力则依赖于前提对于结论的证成质量。因而,一个权衡论证之所以具有其相应的说服力,必定是因为该论证真实对应、并切实展现了一种权衡机制,进而得出了一个无偏见的结论。所以,它在表达形态上的“不完全性”,只可能是论证者所做出的省略。进而,那些未曾言明的内容成分,也即是论证中被省略的前提,它们需要在分析与重构论证时加以补充。于是,这一逻辑重构将反面理由作为前提,并通过补充“平衡考虑前提”来还原出一个权衡机制。然而,纵使抛开之前所提及的那些由此带来的理论问题,这一重构方式本身是否恰当,也仍然是可以商\mbox{榷的。}

论证重构的一条根本原则,是要尽可能忠实于论证者真实做出的那个论证,因而,一个论证在重构时是否需要补充某个省略前提,这取决于论证者是否确实使用了该前提、但又没有对之加以明确表述。但是,论证作为一种交际行为也需要受到一般交际原则的约束,因而,论证者对于前提的省略也不能随意。由此来看,对权衡论证的上述逻辑重构就显现出了某种反常之处。一方面,这一重构方式预设了论证者在其权衡论证中确实是通过对正反面理由加以权衡,进而判定正面理由胜过反面理由,从而最终证明了结论合理。但另一方面,就权衡论证的真实表现形态来看,论证者却只是通过简单的语言手段来构建一种正反面理由之间的不均衡对比,从而通过其“会话含义”来间接暗示出“正面理由胜过反面理由”这一权衡结果,同时,他对于在它们之间进行权衡的方式与过程,却又只字未提。由此,上述逻辑重构方式要能成立,也即意味着论证者明明建构了一个以“权衡”作为特定方式的论证来证明结论,却又在其中故意省略了对与“权衡”相关的前提做出明确表述。但是,这无疑等同于是说论证者在违反最为基本的交际准则(比如格赖斯会话合作原则中的“数量原则”),因为他在表达一个论证时竟然刻意将该论证中所包含的最为重要、最必不可少的信息加以了省略。而且,换个角度来看,这一做法也意味着论证者向其听众展现着某种理智上的轻慢、甚至是不负责任的态度,因为他在明确承认反面理由存在的同时,却又立即在不提供任何所持依据的情况下,就对其价值加以了直接否定。由此而言,上述逻辑重构方式实际上最终导致了一种对论证者的不宽容解读。

更进一步来看,其实权衡论证这一现象正好将逻辑学论证分析进路导入了一个理论困境,从而也将其局限性暴露出来。如果坚持认为权衡论证对应了完全的权衡机制,那么其形态上的不完全性就不可接受,它必须通过省略前提来加以补充重构,但这将导向对论证者本人的不宽容解读;如果转而接受权衡论证在形态上的不完全具有合理性,那么它将无法对应一个合格的权衡机制,进而,反面理由的逻辑功能无法得到恰当解释,权衡论证的说服力也无法再依赖于权衡机制来进行说明。显然,对于非形式逻辑学者而言,这两种情形中的后一种更让人难以接受,因为它意味着权衡论证的说服力不能完全对等于其前提对结论的证成质量,而这正好触及,甚至是突破了逻辑学进路论证分析的理论底线。

实际上,权衡论证的逻辑重构所面临的这些难题,正表明它的解析要求我们突破单纯的逻辑学进路,从而更充分地尊重论证实践的真实和复杂形态,并有效利用不同研究进路的理论洞见。权衡论证中最为本质的特征,应当是论证者以“尽管”“纵使”这类转折性关联词来引导反面理由。实质而言,这类转折性关联词的应用是一种特殊的语言表达手段和修辞技巧。逻辑学家奎因就曾明确意识到,“对于‘但是’、‘尽管’这类语言联结词的考察,为我们揭示出了语言所具有的两个不同层面,一个是语言的逻辑层面,一个是语言的修辞层面”(\parencite{a12},第40页)。换言之,这类转折性关联词更多地展现着语言的修辞效果,若对它们采取一种纯粹逻辑学的解读,这并不恰切,甚至可能是误入歧途。相反,通过将权衡论证分析为一种策略操控的模式,这正是为权衡论证的解析引入了修辞学的视角。这一分析方法弱化了对权衡论证中反面理由的逻辑解读,并转而强调其语言联结词所达到的修辞效果。由此我们才更清晰地揭示出,论证者通过该类转折性关联词来引述反面理由,正是在运用一种特殊的修辞策略,其目的在于通过其会话含义来引导听众去完成对某个权衡机制的理解与重构,并由此来增进其论证对于听众的说服效果。

\vspace{-0.5em}

\section{结语}

\vspace{-0.5em}

当代西方论证理论近半个世纪以来的蓬勃兴盛,归功于不同论证研究进路的全面复兴和持续发展。但时至今日,当前论证研究已经进入对不同理论进路加以反思和整合的全新阶段。对于生活中复杂论证现象的分析,需要全面关注论证行为的不同层面,进而,也要求融合不同研究进路的理论工具和方法。语用论辩学的拓展理论,就对应了一种整合论辩术和修辞学进路的最新理论发展方向,借助其“策略操控”的分析方法,我们可以更好地解析论证实践中的复杂现象。作为一种兼顾“证成结论”和“达成说服”的复杂论证行为,权衡论证中既包含了论证者对其结论所做的合理性证明,也展现着论证者为追求说服效果而运用的修辞技巧,只有整合不同学科进路的分析视角和方法,才能对之加以准确的解析。结合语用论辩学的拓展理论,权衡论证可以被恰当地分析为一种策略操控的特定模式,这一解析方式既能够阐明权衡论证的机制和效果,也具有理论上的清晰性和简洁性。相反,过于强调其中反面理由的逻辑功能,并由之展开其逻辑重构,这既会带来棘手的理论难题,也未能准确刻画权衡论证的最重要特征。


%-------------------------------------------------------------------------%
%    2.8 参考文献
%-------------------------------------------------------------------------%
\vspace{0ex} 
\printbibliography


%-------------------------------------------------------------------------%
%    2.9 打印非原创文章的作者信息(作者请忽略此部分)
%-------------------------------------------------------------------------%
\ifthenelse{\equal{\myarticletype}{original}}{}{%
\vspace*{4ex}
\noindent{\kaishu \myfirstauthor}                                       % 第一作者
{\myfirstaffiliation  }                                                   % 第一作者单位
{\myfirstemail   }                                                          % 第一作者email
% \vspace*{1ex}                                                                %如果需要请取消注释
% {\kaishu \mysecondauthor}                                            % 第二作者
% {\mysecondaffiliation }                                                 % 第二作者单位
%{ \mysecondemail }                                                     % 第二作者email
% \vspace*{1ex}                                                           %如果需要请取消注释
% {\kaishu \mythirdauthor}                                              % 第三作者
% \mythirdaffiliation                                                   % 第三作者单位
% \mythirdemail                                                           % 第三作者email
}

%-------------------------------------------------------------------------%
%    2.10 打印责任编辑(作者请忽略此部分)
%-------------------------------------------------------------------------%
\vspace{-1ex}
\begin{flushright}
\myeditor
\end{flushright}

%-------------------------------------------------------------------------%
%    2.11 根据需要打印英文摘要(作者请忽略此部分)
%-------------------------------------------------------------------------%
\ifthenelse{\equal{\mytitleEN}{null}}{}{%
    \newpage
    \printtitlepageEN
}

\end{document}
