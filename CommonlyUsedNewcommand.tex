%Font Settings
\setCJKsansfont[FakeBold=2]{KaiTi} % In case bold KaiTi is needed.



%cite
%ParenCite Raise
\newcommand{\pcr}[1]{\raise.1em\hbox{[\cite{#1}]}}
\newcommand{\prc}{\pcr} %typo

%Closed PCR
\newcommand{\cpcr}[1]{(\pcr{#1})}
%example 	\cpcr{cite}			->	([3])

%cite+page
\newcommand{\cp}[2]{(\pcr{#1},第 #2 页)}
%example 	\cp{cite}{4}	-> ([3],第 4 页)

%page+cite 
\newcommand{\pc}[2]{\pcr{#2}, #1 }
%example	\pc{第12页}{cite}		->	[3],第12页
%If someone writes (\cite[page]{cite}), then turn it into (\pc{page}{cite})
%use a vim macro or substitute
%  :%s/\\cite\[\(.\{-1,}\)\]/\\pc{\1} 

%dash
\newcommand{\cndash}{\raise0.1em\hbox{--}}



%math
\newcommand{\set}[1]{\{#1\}}
\newcommand{\pair}[1]{\langle #1\rangle}
\newcommand{\pari}{\pair} %typo
\newcommand{\comp}[1]{#1^\mathsf{c}}%{#1^\complement}%

\newcommand{\inv}{^{-1}}

\newcommand{\quo}[1]{#1/{\equiv}} %quotient


\newcommand{\fun}[3]{#1:#2\to #3}		%<FUNcion> from <Domain> to <Codomain>
\newcommand{\homnat} [2]{\Phi_{#1,#2}} %hom nat. trans. in adjunction
\newcommand{\homnati}[2]{\Psi_{#1,#2}} %hom nat. trans. Inverse
\newcommand{\maps}[3]{#1:#2\mapsto #3} %<Function> MAPS <elem_in_dom> to <elem_in_cod>

% Evil Category, which is in contradiction with the def above.
\newcommand{\from}{\leftarrow}
\newcommand{\fun}[3]{#3 \from #2:#1}
\newcommand{\homnat} [2]{{_{#2,#1}}\Phi}
\newcommand{\homnati}[2]{{_{#2,#1}}\Psi}
\newcommand{\maps}[3]{#3\mapsfrom #2:#1}

%Dirty enumerate

\newcommand{\enu}[3]{#1{_1} #3\ldots #3 #1{_{#2}}} 
%if #1 is A  #3 is ,  #2 is i   then it looks like A{_1} ,\ldots , A{_i}
%subscripts are grouped s.t. they can be input into other macros
	\newcommand{\enucm}[2]{\enu{#1}{#2}{,}} %cm for comma
\newcommand{\subt}[3]{#1_{#2#3}} % #3 is left for the {_1} part in def of \enu above. 
	\newcommand{\subenu}[4]{\enu{\subt #1#2}{#3}{#4}} 
	% \subt BAx is B_{Ax}, where x is {_1}.
		\newcommand{\subenucm}[3]{\subenu {#1}#2#3,}
\newcommand{\compsubt}[3]{\comp{#1_{#2#3}}} %similar to \subt
	\newcommand{\compsubenu}[4]{\enu{\compsubt #1#2}{#3}{#4}}
		\newcommand{\compsubenucm}[3]{\enu{\compsubt #1#2}{#3},}

%commands goes with \enu
%representatives in quotient structures 
\newcommand{\rep}[2]{[#1#2]}
%example	\enucm{\rep a}n  	-> [a_1],\ldots,[a_n]
%function in parenthesis
\newcommand{\func}[2]{f(#1#2)}
%example	\enucm{\func a}n 	-> f(a_1),\ldots,f(a_n)




%Environment setting.

\makeatletter
\renewenvironment{quotation}
               {\renewcommand{\baselinestretch}{1.2} \it
               \list{}{\listparindent 2em%
               \rightmargin=0em \leftmargin=2em
                        \itemindent    \listparindent
                        \rightmargin   \leftmargin
                        \parsep        \z@ \@plus\p@}%
                \item\relax}
               {\endlist}
               
\renewenvironment{quote}
               {\renewcommand{\baselinestretch}{1.2} \it
               \list{}{\listparindent 0em%
               \rightmargin=0em \leftmargin=2em
                        \itemindent    \listparindent
                        \rightmargin   \leftmargin
                        \parsep        \z@ \@plus\p@}%
                \item\relax}
               {\endlist}
               
\makeatother 

